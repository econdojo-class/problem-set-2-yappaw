%------------%
%  Preamble  %
%------------%

\documentclass[final,11pt]{article}
\usepackage[paperwidth=9.0in, top=1.2in, bottom=1.2in, left=1.2in, right=1.2in]{geometry}
\usepackage{amsmath}
\usepackage{color}
\usepackage{multirow}
\usepackage{setspace}
\usepackage{fancyhdr}
\usepackage{longtable}
\usepackage{array}
\usepackage{booktabs}
\usepackage{mathpazo}
\usepackage{threeparttable}
\usepackage{eurosym}
\usepackage{graphicx}
\usepackage[colorlinks, linkcolor=blue, anchorcolor=blue, citecolor=blue]{hyperref}

\renewcommand{\headrulewidth}{0pt}
\setlength{\arraycolsep}{10pt}
\setlength\headheight{0.5cm}
\setlength\headsep{0.8cm}
\setlength\footskip{1.0cm}
\setlength{\parindent}{0em}
\pagestyle{fancy}
\chead{\textcolor[rgb]{0.5,0.5,0.5}{\sc Spring 2025: ECON 6100}}

%------------%
%  Document  %
%------------%

\begin{document}
\thispagestyle{empty}
\begin{spacing}{1.25}

\textbf{Your Name: Yaw Appaw \hfill Problem Set 2, Due: Mar. 18, 2025}\\

(1) Use the probability integral transformation method to simulate from the distribution
\begin{gather}
    f(x) = 
    \begin{cases}
        \frac{2}{a^2}x,  & \text{if }0\leq x\leq a \\
        0, & \text{otherwise}
    \end{cases}
\end{gather}
where $a>0$. Set a value for $a$, simulate various sample sizes, and compare results to the true distribution.

\section *{Solution}
We use the cumulative distribution function (CDF):

\[
F(x) =
\int_0^x \frac{2}{a^2} t \, dt = \frac{x^2}{a^2}
\]

The inverse of the CDF is:

\[
x = a \sqrt{u}, \quad u \sim \text{Uniform}(0, 1)
\]
Step 3: Results
The plot below shows the histogram of the simulated data along with the true PDF:

\begin{center}
\includegraphics[width=0.7\textwidth]{Image 1.jpg}
\end{center}


\newpage

(2) Generate samples from the distribution
\begin{gather}
    f(x)=\frac{2}{3}e^{-2x}+2e^{-3x}
\end{gather}
using the finite mixture approach.

\section*{Solution}
Step 1: Mixture Approach
We generate samples from the following components:

\[
\begin{cases}
X_1 \sim \text{Exponential}(\lambda = 2) \\
X_2 \sim \text{Exponential}(\lambda = 3)
\end{cases}
\]

with mixture weights \( w_1 = \frac{2}{3}, w_2 = \frac{1}{3} \).
Step 2: Simulation
First, select the component using the weights.
Then generate a sample from the chosen component.

Step 3: Results
The histogram of the sampled data along with the true PDF is shown below:

\begin{center}
\includegraphics[width=0.7\textwidth]{Picture3.jpg}
\end{center}

\newpage

(3) Draw 500 observations from Beta$(3,3)$ using the accept-reject algorithm. Compute the mean and variance of the sample and compare them to the true values.

\section*{Solution}
Step 1: Accept-Reject Method

Target distribution: \( f(x) = \text{Beta}(3, 3) \)

Proposal distribution: \( g(x) = \text{Uniform}(0, 1) \)

Set the constant:

\[
M = \max_{0 \leq x \leq 1} \text{Beta}(3, 3) = \frac{2^4}{6} = 1.5
\]
1. Draw \( x \sim \text{Uniform}(0, 1) \)  
2. Draw \( u \sim \text{Uniform}(0, M) \)  
3. Accept \( x \) if:

\[
u \leq \text{Beta}(x; 3, 3)
\]

\[
\begin{aligned}
\text{Sample Mean} &= 0.4987, \quad \text{True Mean} = 0.5 \\
\text{Sample Variance} &= 0.0345, \quad \text{True Variance} = 0.0357
\end{aligned}
\]

The simulated mean and variance are very close to the true values, showing that the accept-reject algorithm worked well.

\end{spacing}
\end{document}
